\section{Einleitung}

Die vorliegende Bachelorarbeit hat zunächst als Ziel, die bestehende
Robotersteuerungssoftware von Micro-ROS auf FreeRTOS zu portieren, um die
Echtzeitleistung beider Plattformen miteinander zu vergleichen.

Beide Systeme sind für die Steuerung eines mobilen Roboters auf einem Cortex-M7
Mikrocontroller ausgelegt, unterscheiden sich aber in ihrer zugrundeliegenden
Softwarearchitektur (\ref{fig:micro_ros_arch}): Während Micro-ROS auf dem
\ac{ROS 2} Framework aufbaut und somit eine höhere Abstraktionsebene durch eine
standardisierte Kommunikationsschnittstelle in Form einer integrierten
\ac{DDS}-Middleware bietet, basiert dies selbst auf einem \ac{RTOS} wie
FreeRTOS. Die Portierung auf FreeRTOS kann daher als eine Reduzierung von
Abhängigkeiten betrachtet werden. Dies ermöglicht eine direktere und damit
effizientere Nutzung der zugrundeliegenden Echtzeit- sowie Speicherressourcen.

\begin{figure}[htb] \centering
    \includegraphics[width=0.8\textwidth]{assets/Micro-ROS_architecture}
    \caption{Micro-ROS Architektur\cite[S. 6]{koubaa2023}}
    \label{fig:micro_ros_arch}
\end{figure}

Daher wird die Steuerungssoftware erneut auf Basis von FreeRTOS implementiert,
um die Echtzeitleistung auf beiden Plattformen zu analysieren und zu
vergleichen. Die Analyse soll unter anderem aufzeigen, inwiefern FreeRTOS durch
die Eliminierung dieser zusätzlichen Abhängigkeit eine effizientere und
leichtgewichtigere Lösung darstellt. Dabei soll der Einsatz einer zyklengenauen
Messung des Programmablaufs ermöglicht werden, um fundierte Aussagen über das
Echtzeitverhalten beider Systeme zu treffen und den Leistungsgewinn quantitativ
zu belegen.

Die Arbeit gliedert sich in vier Teile: Nach der Einführung in die grundlegenden
Konzepte wird zunächst die Implementierung der Steuerungssoftware auf FreeRTOS
beschrieben. Anschließend wird die Implementierung zur Erfassung von
Laufzeitdaten detailliert gezeigt. Den Abschluss bildet die quantitative
Präsentation der Ergebnisse, deren Bewertung sowie mögliche Optimierungsansätze.
