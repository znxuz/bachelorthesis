\section{Abschluss}

\subsection{Fazit}

Am Anfang wurde die Robotersteuerungssoftware von Micro-ROS auf FreeRTOS
umgestellt. Anschließend wurden eine Multi-Producer-Senke sowie ein Verfahren
entwickelt, das Echtzeitinformationen über die Steuerungssoftware ausgeben kann.
Abschließend wurde die Echtzeitfähigkeit basierend auf die Echtzeitinformationen
analysiert.

Daraus lässt sich schlussfolgern, dass die Steuerungssoftware zwar durch
Integration von Micro-ROS funktionsreicher und folglich mit einer Vielzahl von
ROS-Software-Komponenten kompatibel wird, dies allerdings mit erheblichem
Overhead erkauft wird. Bei begrenztem Speicher oder Rechenleistung bleibt
FreeRTOS mit seinem schlanken Kernel und den threadsicheren
Queue-Implementierungen weiterhin eine geeignete Wahl gegenüber komplexeren
RTOS-Lösungen -- besonders wenn harte Echtzeitfähigkeit im Vordergrund steht.

\subsection{Ausblick}

Für zukünftige Arbeiten könnte die Multi-Producer-Senke so weiterentwickelt
werden, dass sie atomare Schreiboperationen auf 4-Byte-/32-Bit-Ebene
unterstützt. Dadurch könnten die Echtzeit-Daten nicht mehr im menschenlesbaren
Format, sondern maschinenlesbar und komprimiert als 4-Byte-Dateneinheit
ausgegeben werden.

Dies würde den Zwischenpuffer für die Echtzeitanalyse überflüssig machen, da die
durch Kontextwelchsel/Interrupt generierten Daten direkt atomar in die Senke
geschrieben werden könnten. Dazu müsste ein Parser auf dem Host-System
entwickelt werden, der idealerweise auch die Visualisierung sowie Analyse der
Daten parallel in Echtzeit ermöglicht.
