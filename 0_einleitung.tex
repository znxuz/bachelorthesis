\section*{Einleitung}

Die vorliegende Arbeit beschäftigt sich mit der Analyse der Echtzeitfähigkeit
von Micro-ROS und FreeRTOS am Beispiel einer Robotersteuerungssoftware. Ziel ist
es, die Performance beider Systeme hinsichtlich der Ausführungszeiten,
Ressourcenverbrauch sowie Echtzeitverhalten zu untersuchen, um ihre Eignung für
Roboteranwendungen zu bewerten.

Die Arbeit beinhaltet schwerpunktmäßig die Entwicklung einer Methode zum
Profiling der Steuerungssoftware eines mobilen Roboters. Dabei wird zunächst die
bestehende Firmware, die auf Micro-ROS basiert, im Rahmen dieser Arbeit auf
FreeRTOS portiert. Anschließend wird die Methodik zur Generierung von
Profiling-Daten für die Analyse festgelegt und implementiert, und die
resultierende Ergebnisse evaluiert.

Zu Beginn wird ein Überblick über die grundlegenden Konzepte gegeben.
Darauffolgend werden die Implementierungen detailliert beschrieben. Abschließend
werden die erzielten Ergebnisse vorgestellt und bewertet, und es wird ein
Ausblick auf weitere Anwendungsmöglichkeiten und Optimierungspotenziale gegeben.
