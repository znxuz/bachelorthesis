\section*{Kurzfassung}

Diese Arbeit analysiert die Echtzeitfähigkeit von Micro-ROS und FreeRTOS am
Beispiel einer Robotersteuerungssoftware. Ziel ist es, die Performance beider
Systeme im Hinblick auf Ausführungszeiten, Ressourcenverbrauch und
Echtzeitverhalten zu vergleichen.

Die Analyse beginnt zuerst mit der vollständigen Umstellung der bestehenden
Robotersteuerungssoftware von Micro-ROS auf FreeRTOS. Anschließend wird die Data
Watchpoint and Trace Unit (DWT) Zur Analyse eingesetzt, um eine zyklengetreue
Erfassung des Programmlaufs zu ermöglichen.

Abschließend wird das Ergebnis evaluiert, welches unter anderem die
Ausführungszeiten von FreeRTOS-Prozessen, zeitkritischen Funktionen sowie das
Verhältnis von Ausführung zu Leerlaufzeit umfasst. Die Ergebnisse sollen
Einsichten darüber geben, inwieweit Micro-ROS und FreeRTOS für
Echtzeitanwendungen in der Robotik geeignet sind und welche Vor- oder Nachteile
die jeweiligen Systeme bieten.

\section*{Abstract}
