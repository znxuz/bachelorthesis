\section*{Einleitung}

Die vorliegende Bachelorarbeit hat zunächst als Ziel, die bestehende
Robotersteuerungssoftware von Micro-ROS auf FreeRTOS zu portieren, um die
Echtzeitleistung beider Plattformen zu analysieren und miteinander zu
vergleichen.

Beide Systeme sind für die Steuerung eines mobilen Roboters auf einem Cortex-M7
Mikrocontroller von Arm entwickelt, unterscheiden sich jedoch in ihrer
grundlegenden Architektur, was sich auch in ihrer Echtzeitfähigkeit und
Ressourcennutzung widerspiegelt. Während Micro-ROS auf dem \ac{ROS 2} Framework
aufbaut und eine höhere Abstraktionsebene sowie standardisierte
Kommunikationsschnittstellen mittels der integrierten \ac{DDS}-Middleware
bietet, basiert dies selbst auf FreeRTOS. Die Portierung auf FreeRTOS kann daher
als eine Reduzierung der Abhängigkeitsebene betrachtet werden. Dies ermöglicht
eine direktere und effizientere Nutzung der zugrunde liegenden Echtzeit-, sowie
Speicherressourcen.

\begin{figure}[htb] \centering
    \includegraphics[width=0.8\textwidth]{assets/Micro-ROS_architecture}
    \caption{Micro-ROS Architektur\cite[S. 6]{koubaa2023}}
\end{figure}

Nach der Portierung auf FreeRTOS wird die Echtzeitleistung der
Steuerungssoftware analysiert, wobei insbesondere die Ausführungsdauer
zeitkritischer Funktionen sowie Tasks untersucht wird. Der Vergleich soll unter
anderem aufzeigen, inwiefern FreeRTOS durch die Eliminierung dieser zusätzlichen
Abhängigkeit eine effizientere und leichtgewichtige Lösung für kritische
Roboteranwendungen darstellt. Dabei soll der Einsatz einer zyklengenauen Messung
des Programmablaufs ermöglicht werden, um fundierte Aussagen über die
Echtzeitfähigkeit  beider Plattformen zu treffen, und den Leistungsgewinn anhand
dieses  Beispiels für eine Steuerungssoftware quantitativ zu belegen.

Die Arbeit gliedert sich in drei Hauptteile: Nach einer Einführung in die
grundlegenden Konzepte folgt eine detaillierte Beschreibung der Implementierung
des FreeRTOS-Systems sowie des Verfahrens zur Echtzeitanalyse. Den Abschluss
bildet die Präsentation der Ergebnisse, deren Bewertung sowie mögliche
Optimierungsansätze.
