\section{Abschluss}

Am Anfang wurde die Robotersteuerungssoftware von Micro-ROS auf FreeRTOS
umgestellt. Anschließend wurden eine Multi-Producer-Senke sowie ein Verfahren
entwickelt, das Echtzeitinformationen über die Steuerungssoftware ausgeben kann.
Abschließend wurde die Echtzeitfähigkeit basierend auf die Echtzeitinformationen
analysiert.

\subsection{Fazit}

Es lässt sich schlussfolgern, dass die Steuerungssoftware zwar durch Integration
von Micro-ROS funktionsreicher und folglich mit einer Vielzahl von
ROS-Komponenten kompatibel wird, dies allerdings mit erheblichem Overhead
erkauft wird. Bei begrenztem Speicher oder Rechenleistung bleibt FreeRTOS mit
seinem schlanken Kernel und den threadsicheren Queue-Implementierungen weiterhin
eine geeignete Wahl gegenüber komplexeren RTOS-Lösungen -- besonders wenn harte
Echtzeitfähigkeit im Vordergrund steht.

Darüber hinaus wurde gezeigt, dass sich die Softwareleistung durch L1-Caches
deutlich verbessern lässt, was für leistungskritische Software entscheidend ist.

\subsection{Ausblick}

Für zukünftige Arbeiten könnte die Multi-Producer-Senke so weiterentwickelt
werden, dass sie atomare Schreiboperationen auf 4-Byte-/32-Bit-Ebene
unterstützt. Dadurch könnten die Echtzeitdaten nicht mehr im menschenlesbaren
Format, sondern maschinenlesbar und komprimiert jeweils als 4-Byte-Dateneinheit
ausgegeben werden.

Dies würde erstens den Zwischenpuffer zum Speichern von durch
Kontextwelchsel/Interrupt generierten Zyklenstempeln überflüssig machen, da sie
als 32-bit Daten atomar direkt in die Senke geschrieben werden könnten. Zweitens
könnte ein Parser auf dem Host-System entwickelt werden, der idealerweise auch
die Visualisierung sowie Analyse der Daten parallel in Echtzeit ermöglicht.
Diese Vorgehensweise ähnelt der Rust-Bibliothek \mintinline{text}|defmt|, bei
der die rechenintensive Formatierung von Logging-Informationen nicht auf dem
ressourcenbeschränkten System erfolgt, sondern auf eine sekundäre Maschine
ausgelagert wird.
