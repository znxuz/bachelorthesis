\section{Zusammenfassung}

Am Anfang wurde die Robotersteuerungssoftware von Micro-ROS auf FreeRTOS
umgestellt. Anschließend wurden eine Multi-Producer-Senke sowie ein Verfahren
entwickelt, das Laufzeitinformationen über die Steuerungssoftware ausgeben kann.
Abschließend wurde die Echtzeitfähigkeit basierend auf die erzeugten
Laufzeitdaten analysiert.

Es lässt sich schlussfolgern, dass die Steuerungssoftware zwar durch Integration
von Micro-ROS funktionsreicher und folglich mit einer Vielzahl von
ROS-Komponenten kompatibel wird, dies allerdings mit erheblichem Overhead
erkauft wird. Bei begrenztem Speicher oder Rechenleistung bleibt FreeRTOS mit
seinem schlanken Kernel und den standardmäßig threadsicheren Queue-Abstraktionen
weiterhin eine geeignete Wahl gegenüber komplexeren RTOS-Lösungen --
insbesondere wenn hochfrequente Aufgabenausführungen mit Echtzeitanforderungen
prioritär sind.

Zudem wurde demonstriert, dass L1-Caches die Performance signifikant steigern --
eine für leistungskritische Software wesentliche Optimierung, die nicht
vernachlässigt werden darf.

Für zukünftige Arbeiten könnte die Multi-Producer-Senke so weiterentwickelt
werden, dass sie Schreiboperationen atomar auf 4-Byte-/32-Bit-Ebene unterstützt.
Dadurch könnten die Laufzeitdaten nicht mehr im menschenlesbaren Format mit
überflüssigem Daten-Overhead, sondern komprimiert jeweils als 32-Bit-Datenwort
ausgegeben werden.

Dies würde erstens den Zwischenpuffer und den erforderlichen Mutex zum Speichern
sowie Serialisieren von erzeugten Laufzeitdaten überflüssig machen, da sie als
32-bit-Datenworte auf der ARM-Architektur atomar geschrieben werden könnten.
Zweitens könnte dann ein dedizierter Parser auf dem Linux-Host entwickelt
werden, der die Visualisierung übernimmt und idealerweise eine Echtzeitanalyse
parallel zur Laufzeit durchführt.

% Diese Vorgehensweise ähnelt der Rust-Bibliothek \mintinline{text}|defmt|, bei
% der die rechenintensive Formatierung von Logging-Informationen nicht auf dem
% ressourcenbeschränkten System erfolgt, sondern auf eine sekundäre Maschine
% ausgelagert wird.
