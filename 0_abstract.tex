\section*{Kurzfassung}

Diese Arbeit analysiert die Echtzeitfähigkeit von Micro-ROS und FreeRTOS am
Beispiel einer Robotersteuerungssoftware. Ziel ist es, beide Systeme
hinsichtlich ihres Echtzeitverhaltens zu veranschaulichen und zu vergleichen.

Zunächst wird die bestehende Robotersteuerung vollständig von Micro-ROS auf
FreeRTOS portiert. Anschließend wird ein Verfahren zur zyklengenauen Erfassung
des Programmlaufs entwickelt.

Abschließend werden die Ergebnisse ausgewertet, darunter Ausführungszeiten von
Tasks und zeitkritischen Funktionen sowie das Verhältnis von Berechnungs- zu
Leerlaufzeiten des Prozessors. Die Resultate zeigen, wie gut sich Micro-ROS und
FreeRTOS für Echtzeitanwendungen in der Robotik eignen und welche Vor- und
Nachteile sie bieten.

\section*{Abstract}

This work analyzes the real-time capabilities of Micro-ROS and FreeRTOS using a
robot control software as an example. The goal is to visualize and compare both
systems regarding their real-time behavior.

First, the existing robot control software is fully ported from Micro-ROS to
FreeRTOS. Subsequently, a method for cycle-accurate tracing of program execution
is developed.

Finally, the results are evaluated, including execution times of FreeRTOS tasks,
time-critical functions, and the ratio of computation to idle time of the
processor. The outcomes demonstrate how suitable Micro-ROS and FreeRTOS are for
real-time applications in robotics and what advantages and disadvantages each
system offers.
