\section{Implementierung zur Echtzeitanalyse}

Nachdem die Steuerungssoftware auf zwei verschiedenen Architekturen, nämlich
FreeRTOS und Micro-ROS, aufgebaut wurde, kann nun die Implementierung einer
konkreten, threadsicheren Methode zur Echtzeitanalyse erfolgen. Ziel der Analyse
ist es, Informationen darüber zu gewinnen, wie lange ein bestimmter Task oder
eine bestimmte zeitkritische Funktion benötigt. Die daraus resultierenden Daten
müssen mit einer angemessenen Genauigkeit erfasst werden, um sicherzustellen,
dass die Echtzeitaspekte korrekt widergespiegelt werden.

% TODO uart streambuf impl

\subsection{Aktivierung der DWT}

Wie im vorherigen Abschnitt erläutert \ref{sec:dwt}, stellt die DWT einen
geeigneten Ansatz zur Generierung von Analysedaten dar. Sie ist standardmäßig
auf Cortex-M7-Prozessoren verfügbar und kann durch die folgenden
Konfigurationsschritte aktiviert werden:

\begin{code}
\begin{minted}{cpp}
void enable_dwt() {
  CoreDebug->DEMCR |= CoreDebug_DEMCR_TRCENA_Msk;
  DWT->LAR = 0xC5ACCE55;  // software unlock
  DWT->CYCCNT = 1;
  DWT->CTRL |= DWT_CTRL_CYCCNTENA_Msk;
}
\end{minted}
    \captionof{listing}{Aktivierung der DWT \cite{StackOverflow_DWT_Activation}}
\end{code}
