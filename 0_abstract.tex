\section*{Kurzfassung}

Diese Arbeit analysiert die Echtzeitfähigkeit von Micro-ROS und FreeRTOS anhand
einer Robotersteuerungssoftware. Ziel ist der Vergleich beider Systeme
hinsichtlich ihres Echtzeitverhaltens mit Schwerpunkt auf den Ausführungszeiten.
Dazu wird zunächst die bestehende Micro-ROS-Steuerungssoftware auf FreeRTOS
portiert. Anschließend entwickelt die Arbeit eine zyklengenaue Messmethode für
den Programmlauf, basierend auf der Data Watchpoint and Trace Unit (DWT) der
ARM-Architektur, um Laufzeitdaten zu erfassen. Zum Abschluss verdeutlicht die
Auswertung der Ergebnisse die unterschiedlichen Schwerpunkte beider Plattformen:
Micro-ROS punktet durch die Kompatibilität mit dem ROS-Ökosystem, während
FreeRTOS mit minimalen Latenzen und deterministischem Scheduling deutlich
besseres Echtzeitverhalten bietet. Die Analyse bestätigt zudem den maßgeblichen
Einfluss der L1-Cache-Nutzung auf die Rechenleistung.

\section*{Abstract}

This thesis analyzes the real-time capabilities of Micro-ROS and FreeRTOS using
a robotic control software as a case study. The goal is to compare both systems
regarding their real-time behavior emphasized on the execution times. First, the
existing Micro-ROS robotic control software is ported to FreeRTOS. Next, a
cycle-accurate measurement method for program execution is developed, based on
ARM's Data Watchpoint and Trace Unit (DWT), to capture real-time data at
run-time. The evaluation of the results highlights the different strengths of
both platforms: Micro-ROS provides compatibility to the ROS ecosystem, while
FreeRTOS achieves superior real-time behavior through minimal latencies and
deterministic scheduling. Additionally, the analysis confirms the significant
impact of L1 cache utilization on system performance.
