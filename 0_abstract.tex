\section*{Kurzfassung}

Diese Arbeit analysiert die Echtzeitfähigkeit von Micro-ROS und FreeRTOS anhand
einer Robotersteuerungssoftware. Ziel ist der Vergleich beider Systeme
hinsichtlich ihres Echtzeitverhaltens mit Schwerpunkt auf den Ausführungszeiten.
Dazu wird zunächst die bestehende Micro-ROS-Steuerungssoftware auf FreeRTOS
portiert. Anschließend wird eine zyklengenaue Messmethode für den Programmlauf
basierend auf der Data Watchpoint and Trace Unit (DWT) der ARM-Architektur
entwickelt. Zum Abschluss werden die Laufzeitdaten visualisiert und ausgewertet.
Die Ergebnisse verdeutlichen die unterschiedlichen Schwerpunkte beider
Plattformen: Micro-ROS punktet durch die Kompatibilität mit dem ROS-Ökosystem,
während FreeRTOS mit minimalen Latenzen und deterministischem Scheduling
deutlich besseres Echtzeitverhalten bietet. Die Analyse bestätigt zudem den
maßgeblichen Einfluss der Cache-Nutzung auf die Rechenleistung.

\section*{Abstract}

This thesis analyzes the real-time capabilities of micro-ROS and FreeRTOS using
a robotic control software. The goal is to compare both systems regarding their
real-time behavior emphasized on the execution times. First, the existing
micro-ROS control software is ported to FreeRTOS. Next, a cycle-accurate
measurement method for program execution based on ARM's Data Watchpoint and
Trace Unit (DWT)is developed. Finally, the runtime data is evaluated. The
evaluation of the results highlights the different strengths of both platforms:
micro-ROS stands out due to its compatibility with the ROS ecosystem, while
FreeRTOS achieves superior real-time behavior through minimal latencies and
deterministic scheduling. Additionally, the analysis confirms the significant
impact of the cache utilization on system performance.
