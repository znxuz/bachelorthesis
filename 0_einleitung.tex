\section*{Einleitung}

Die vorliegende Bachelorarbeit hat zunächst als Ziel, die bestehende
Robotersteuerungssoftware von Micro-ROS auf FreeRTOS zu portieren, um die
Echtzeitleistung beider Plattformen zu analysieren und miteinander zu
vergleichen.

Beide Systeme sind für die Steuerung eines mobilen Roboters auf einem Cortex-M7
Mikrocontroller von Arm ausgelegt, unterscheiden sich aber in ihrer
zugrundeliegenden Softwarearchitektur: Während Micro-ROS auf dem \ac{ROS 2}
Framework aufbaut und eine höhere Abstraktionsebene sowie standardisierte
Kommunikationsschnittstellen mittels einer integrierten \ac{DDS}-Middleware
bietet, basiert dies selbst auf einem \ac{RTOS} wie FreeRTOS. Die Portierung auf
FreeRTOS kann daher als eine Reduzierung einer Abhängigkeitsebene betrachtet
werden. Dies ermöglicht eine direktere und damit effizientere Nutzung der
zugrundeliegenden Echtzeit-, sowie Speicherressourcen.

\begin{figure}[htb] \centering
    \includegraphics[width=0.8\textwidth]{assets/Micro-ROS_architecture}
    \caption{Micro-ROS Architektur\cite[S. 6]{koubaa2023}}
\end{figure}

Danach wird die Echtzeitleistung der Steuerungssoftware basierend auf den beiden
Plattformen analysiert, wobei die Ausführungsdauer zeitkritischer Funktionen
sowie Tasks untersucht wird. Der Vergleich soll unter anderem aufzeigen,
inwiefern FreeRTOS durch die Eliminierung dieser zusätzlichen Abhängigkeit eine
effizientere und leichtgewichtigere Lösung darstellt. Dabei soll der Einsatz
einer zyklengenauen Messung des Programmablaufs ermöglicht werden, um fundierte
Aussagen über das Echtzeitverhalten beider Plattformen zu treffen und den
Leistungsgewinn quantitativ zu belegen.

Die Arbeit gliedert sich in vier Hauptteile: Nach einer Einführung in die
grundlegenden Konzepte wird zunächst die Implementierung der Steuerungssoftware
auf FreeRTOS beschrieben. Anschließend wird Implementierung zur Erfassung von
Laufzeitdaten detailliert dargestellt. Den Abschluss bildet die Präsentation der
Ergebnisse, deren Bewertung sowie mögliche Optimierungsansätze.
